\section{Template Attacks} % Section Title

%---------------------------------------------------------
% Slide 1: Introduction to Profiled SCAs
\begin{frame}
    \frametitle{Profiled Side-Channel Attacks (SCAs)}
    \begin{block}{Overview}
        \begin{itemize}
            \item A powerful class of SCAs designed to attack targets with minimal measurements (even down to a single one).
            \item Overcome countermeasures that restrict the number of measurements an attacker can acquire.
            \item Involve two main phases:
                \begin{enumerate}
                    \item \textbf{Profiling Phase:} Attacker builds an accurate model of the device's side-channel behavior.
                    \item \textbf{Exploitation Phase:} The model is used to extract the unknown key from the target device.
                \end{enumerate}
        \end{itemize}
    \end{block}
\end{frame}

%---------------------------------------------------------
% Slide 2: Profiling Advantages & Core Idea 
\begin{frame}
    \frametitle{Profiled SCAs: The Template Attack Advantage} % Improved Title
    \begin{block}{The Profiling Advantage}
        \begin{itemize}
            \item Performed on a device \textit{identical} to the target, but fully controlled by the attacker.
            \item This allows setting the secret key value at will and performing unlimited measurements.
            \item Enables deriving a "perfect" side-channel behavior or "profile" for every possible value of the key bits.
        \end{itemize}
    \end{block}
    \begin{block}{Core Idea of Template Attacks}
        \begin{itemize}
            \item TAs are considered the \textbf{most effective profiled attack }in an information-theoretic sense.
            \item They can, in principle, defeat masking and hiding countermeasures if enough measurements are collected during profiling.
        \end{itemize}
    \end{block}
\end{frame}

%---------------------------------------------------------
% Slide 3: Template Attacks: Statistical Modeling 
\begin{frame}
    \frametitle{Template Attacks: Statistical Modeling} % Improved Title
    \begin{block}{Modeling Side-Channel Traces}
        \begin{itemize}
            \item Each side-channel measurement trace is assumed to be affected by \textbf{additive Gaussian noise}.
            \item Each trace $\hat{T}^{(k_i)}$ (for a specific key bit $k_i$) is modeled as a random vector variable $T^{(k_i)}$.
            \item $T^{(k_i)}$ is assumed to follow a \textbf{multivariate Gaussian distribution}: $\mathcal{N}(\mu^{(k_i)}, \Sigma^{(k_i)})$.
                \begin{itemize}
                    \item $\mu^{(k_i)}$: Mean vector (average trace for key $k_i$).
                    \item $\Sigma^{(k_i)}$: Covariance matrix (describes how data points vary together for key $k_i$).
                \end{itemize}
            \item The mean vector ($\mu$) and covariance matrix ($\Sigma$) are computed over selected \textbf{points of interest} (POIs) in the trace, where significant leakage occurs.
        \end{itemize}
    \end{block}
\end{frame}

%---------------------------------------------------------
% Slide 4: Template Attack: Multivariate Gaussian Probability Density Function
\begin{frame}
    \frametitle{Template Attack: Multivariate Gaussian PDF}
    \begin{block}{The Probability Density Function (PDF)}
        For a trace $x$ (composed of selected POIs) belonging to a key hypothesis $k_i$, its probability density function is given by:
        $$
        \text{Pr}(T^{(k_i)}=x) = \frac{\exp\left(-\frac{1}{2}(x-\mu^{(k_i)})^T(\Sigma^{(k_i)})^{-1}(x-\mu^{(k_i)})\right)}{\sqrt{(2\pi)^n \det(\Sigma^{(k_i)})}}
        $$
        \begin{itemize}
            \item $n$: Number of selected points of interest (dimension of the trace vector $x$).
            \item $\mu^{(k_i)}$: Mean vector for key hypothesis $k_i$.
            \item $\Sigma^{(k_i)}$: Covariance matrix for key hypothesis $k_i$.
        \end{itemize}
    \end{block}
\end{frame}

%---------------------------------------------------------
% Slide 5: Template Attack: Profiling Phase - Building Templates 
\begin{frame}
    \frametitle{Template Attack: Profiling Phase - Building Templates} 
    \begin{block}{Generating Templates}
        \begin{itemize}
            \item The attacker, on their controlled device, sets a specific key bit value $k_i$ (e.g., $k_i=0$ or $k_i=1$).
            \item For each possible value of $k_i$, a large number of side-channel traces are collected (from the POIs).
            \item From these measurements, \textit{sample estimates} of the mean vector ($\hat{\mu}^{(k_i)}$) and covariance matrix ($\hat{\Sigma}^{(k_i)}$) are derived.
            \item These statistical models ($\hat{\mu}^{(k_i)}, \hat{\Sigma}^{(k_i)}$) form the \textbf{templates} for each key bit value.
        \end{itemize}
    \end{block}
\end{frame}

%---------------------------------------------------------
% Slide 6: Template Attack: Profiling Phase
\begin{frame}
    \frametitle{Template Attack: Profiling Phase}
    \begin{block}{Example: Targeting a Key Portion}
        \begin{itemize}
            \item If targeting an 8-bit key byte, an attacker might build $2^8 = 256$ templates (one for each possible byte value).
            \item Alternatively, using a divide-and-conquer approach, they could target each bit individually, building two templates (for '0' and '1') for each bit.
        \end{itemize}
    \end{block}
\end{frame}

%---------------------------------------------------------
% Slide 7: Template Attack: The Exploitation Phase
\begin{frame}
    \frametitle{Temlate Attacks: Matching Phase}
    \begin{block}{Key Recovery Steps}
        \begin{itemize}
            \item A single (or few) side-channel trace(s) $\hat{T}$ is acquired from the target device, where the key value is unknown
            \item Measurement conditions must be identical to the profiling phase
            \item For each possible key bit value, the likelihood of the acquired trace $\hat{T}$ belonging to that template's distribution is evaluated
            \item This is done by calculating the \textbf{a-posteriori probability} $\text{Pr}(k_i | \hat{T})$, typically using Bayes' theorem:
            $$
            \text{Pr}(k_i | \hat{T}) = \frac{\text{Pr}(\hat{T} | k_i) \cdot \text{Pr}(k_i)}{\sum_j \text{Pr}(\hat{T} | k_j) \cdot \text{Pr}(k_j)}
            $$
            \item The key bit value $k_i$ that maximizes this a-posteriori probability is selected as the most likely secret key bit.
        \end{itemize}
    \end{block}
\end{frame}
\begin{frame}
\frametitle{Template Attacks}
    \begin{block}{Statistical Efficacy}
        \begin{itemize}
            \item The assumption of multivariate Gaussian distributions for power consumption has proven to be a very good practical approximation.
            \item Statistical tests are used to determine the "goodness of fit" of a model to the actual device behavior, accounting for noise.
        \end{itemize}
    \end{block}
\end{frame}