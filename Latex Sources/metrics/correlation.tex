\section{Correlation as a Metric}

\begin{frame}
    \frametitle{Correlation as a Metric: Motivation}

    \begin{block}{Why Correlation?}
        Most side-channel metrics (SR, GE, score, rank) require success in the attack, but provide little information on how many traces are theoretically needed for success, especially if the attack fails.
    \end{block}
    \begin{itemize}
        \item Correlation gives a way to estimate and project the effort needed to recover the key.
        \item It enables "security projections": estimating how many traces are needed for the correct key candidate to distinguish itself from others.
        \item This is particularly useful when the attack hasn't succeeded yet, allowing for predictive evaluation.
    \end{itemize}
\end{frame}

\begin{frame}
    \frametitle{Pearson Correlation and Trace Projection}

    \begin{block}{Pearson's Correlation in Side-Channel Analysis}
        In CPA (Correlation Power Analysis), Pearson's correlation coefficient $\rho$ is computed between:
        \begin{itemize}
            \item The observed leakage traces $l_1,l_2,\dots,l_n$
            \item The modeled leakages $g(f(x_i,k))$ for a given key candidate $k$
        \end{itemize}
        For the correct key $k_c$ denote this as $\rho_{k_c}$
    \end{block}
    $$ 
    \rho_{k_c} = \frac{
        \sum_{i=1}^n (l_i - \frac{1}{n}\sum_{j=1}^{n}l_j)
        (v_i - \frac{1}{n}\sum_{i=j}^{n}v_j)
     }{
        \sqrt{\sum_{i=1}^n (l_i - \frac{1}{n}\sum_{j=1}^{n}l_j)^2}
        \sqrt{\sum_{i=1}^n (v_i - \frac{1}{n}\sum_{j=1}^{n}v_j)^2}
     }
     $$
     \scriptsize where $v_i = g(f(x_i,k_c))$
\end{frame}


\begin{frame}
    \frametitle{When Does CPA Succeed?}

    \begin{block}{Success Criterion}
        A CPA attack succeeds if $|\rho_{k_c}|$ for the correct key is noticeably higher than the absolute correlations for all incorrect key candidates.
    \end{block}

    \begin{itemize}
        \item A common practical and theoretical assumption: for all incorrect keys, the expected correlation is close to zero.
        \item Thus, success is reduced to distinguishing $\rho_{k_c}$ from zero (not from other possible nonzero values).
    \end{itemize}

    \begin{block}{Why is this Useful?}
        If you know (or can estimate) the "true" value of $\rho_{k_c}$ post-attack, you can predict how many traces $n$ are required for reliable key recovery.
    \end{block}
\end{frame}

\begin{frame}
    \frametitle{Mangard's Shortcut: Projecting the Number of Traces}

    \begin{block}{Formula for the Projected Number of Traces}
        Given an estimate of $\rho_{k_c}$, the formula for predicting the number of traces needed for successful CPA key recovery is:
        \[
        n' = 3 + 8
        \left|
            \frac{z_{\alpha}}{ \ln \left( \frac{1+\rho_{k_c}}{1-\rho_{k_c}} \right) }
        \right|^2
        \]
        where:
        \begin{itemize}
            \item $z_\alpha$ is the z-score for desired type I error $\alpha$ (e.g., $z_{0.01}$ for $1\%$ error);
            \item $\rho_{k_c}$ is the estimated correlation for the correct key hypothesis.
        \end{itemize}
    \end{block}

    
            As $\rho_{k_c}$ decreases, the required number of traces grows quadratically. \newline
             Only the correct key's correlation needs to be estimated: no need to compute all candidate correlations.
      
\end{frame}

%%%%%%%%%%%%%%%%%%%%%%%%%%%%%%%%%%%%%%%%%%%%%%
%%%% POTREI AGGIUNGERE UNA SLIDE SU COME SCEGLIERE z_alpha (in particolare alpha che sta tra 10^-1 e 10^-4
%%%%%%%%%%%%%%%%%%%%%%%%%%%%%%%%%%%%%%%%%%%%%%%

% Correlation and SR

\begin{frame}
    \frametitle{Correlation and Success Rate (SR) Projection}

        Instead of just estimating the number of traces needed for a break, we can predict the \textbf{Success Rate (SR)} of a CPA attack for a given number of traces.
        
        This method, developed by Rivain, uses the statistical properties of correlation scores $\rho$ for all key candidates.
    
    \begin{block}{The Custom Correlation Coefficient}
        The analysis uses a slightly different correlation coefficient, $\ddot{\rho}_k$, for each key candidate $k \in K$:
        $$ \ddot{\rho}_k = \frac{1}{n} \sum_{i=1}^{n} g(f(x_i, k)) \cdot l_i $$
        The key candidate $k$ that maximizes this coefficient is chosen as the best guess by the CPA attack.
    \end{block}
\end{frame}

\begin{frame}
    \frametitle{Statistical Model of Correlation Scores}
    
    \begin{block}{A Multivariate Normal Distribution}
        The core insight is that the vector of all correlation scores, $\ddot{\rho} = [\ddot{\rho}_0, \ddot{\rho}_1, \dots, \ddot{\rho}_{|K|-1}]$, follows a \textbf{multivariate normal distribution}.
        
        This distribution is defined by a mean vector $\mu_{\ddot{\rho}}$ and a covariance matrix $\Sigma_{\ddot{\rho}}$.
    \end{block}
    
            The mean $\mu_k$ for each key candidate $k$ is:
        $$ \mu_k = \frac{1}{|X|} \sum_{x \in X} g(f(x, k) \cdot \hat{\mu}_v$$  \text{where } $v = f(x, k_c)$ 
        
        The covariance $\Sigma_{ij}$ between candidates $i$ and $j$ is:
       $$ \Sigma_{ij} = \frac{1}{n'|X|} \sum_{x \in X} g(f(x, i)) \cdot g(f(x, j)) \cdot \hat{\sigma}^2_v \quad \text{where } v = f(x, k_c) $$
        Here, $\hat{\mu}_v$ and $\hat{\sigma}^2_v$ are the mean and variance of the leakage for a given intermediate value $v$, estimated from experimental traces.
\end{frame}

\begin{frame}
    \frametitle{Projecting SR: The Methodology}
    
        The core of the method is to model the attack's outcome statistically.
        \begin{itemize}
            \item First, the leakage statistics ($\hat{\mu}_v$ and $\hat{\sigma}^2_v$) are estimated using an experimental set of traces.
            
            \item A \textbf{projected number of traces}, $n'$, is chosen. This value, which represents the strength of a hypothetical adversary, is used to define the covariance matrix $\Sigma_{\ddot{\rho}}$.
            
            \item Many random samples are then drawn from the resulting multivariate normal distribution, $\mathcal{N}(\mu_{\ddot{\rho}}, \Sigma_{\ddot{\rho}})$.
            
            \item Each sample vector represents a simulated attack outcome. It is sorted to find the rank of the correct key, $rank_{k_c}$.
            
            \item The \textbf{projected Success Rate (SR)} is the fraction of simulations where $rank_{k_c} = 1$. This can also be extended to find the projected $SR_o$.
        \end{itemize}
    
\end{frame}

\begin{frame}
    \frametitle{Limitations of Correlation-Based Projections}
    
    \begin{block}{ }
        \begin{itemize}
            \item  The projection formulas are based on \textbf{simplified statistical models}. The behavior of real hardware can deviate from these assumptions, due to electrical and electronic effects.
            
            \item \textbf{Estimation errors:} Imperfect statistical estimation of the correlation coefficient reduces the reliability of the results.
            
            \item \textbf{The purpose is Projection:} These are bounds designed to estimate effort when an attack has not yet succeeded, not to guarantee a break.
        \end{itemize}
    \end{block}
\end{frame}
