\section{Number of Traces}
%---------------------------------------------------------
% Slide 1: Number of Traces (N.O.T.): Definition
\begin{frame}
    \frametitle{Number of Traces (N.O.T.): Definition}
    \begin{block}{What it Measures}
        \begin{itemize}
            \item The most commonly used metric to quantify the efficiency of a side-channel attack
            \item It represents the minimum number of side-channel measurements (traces) an attacker needs to successfully recover the secret key
        \end{itemize}
    \end{block}

\newline
    Its primary purpose is to assess how "data-hungry" an attack is. \newline \newline
    A lower N.O.T. indicates a more effective or faster attack, as fewer measurements are required to compromise the secret. \newline \newline
       
        
    
\end{frame}

%---------------------------------------------------------
% Slide 2: Number of Traces (N.O.T.): Limitations
\begin{frame}
    \frametitle{Number of Traces (N.O.T.): Limitations}
    \begin{block}{Challenges and Limitations}
        \begin{itemize}
            \item \textbf{Attack Dependent:} The N.O.T. value is inherently tied to the specific side-channel attack method employed (e.g., DPA, CPA, Template Attacks), making direct comparisons between different attacks difficult
            \item \textbf{Vague Key Recovery:} The term "key recovery" can be imprecise; it might refer to recovering the full key, a partial key, or simply reducing the entropy of the key space, leading to ambiguity in evaluation
            \item \textbf{Conflates Device and Attacker:} N.O.T. is not an inherent measure of a device's security. It conflates the device's inherent physical leakage with the attacker's computational capabilities (e.g., ability to collect many traces, search the key space). This coalescence makes it difficult to obtain a clear picture of the device's true vulnerability and to identify countermeasures that are robust against any attacker

        \end{itemize}
    \end{block}
\end{frame}