\section{Feature selection}
 
\begin{frame}
    \frametitle{Feature Selection: A Crucial Preprocessing Step}
    \begin{block}{Why Feature Selection is Needed}
        \begin{itemize}
            \item Side-channel traces can contain hundreds or thousands of samples (time points)
            \item Too many samples can introduce noise, increase computational complexity, and lead to numerical instability, especially for machine learning algorithms and covariance matrix inversion in Template Attacks
        \end{itemize}
    \end{block}
\end{frame}
\begin{frame}
\frametitle{Feature Selection: Common Techniques}
            \item Feature selection aims to select a subset of trace samples that contain the most sensitive leakage information, often explicitly identifying them as \textbf{Points of Interest (POIs)}, or to transform the data into a smaller, more meaningful set of \textbf{features}
            \item This process helps to filter out redundant or uncorrelated information, making the profiling and exploitation phases more efficient and effective.
    \begin{block}{}
        \begin{itemize}
            \begin{itemize}
                \item \textbf{Maximum Variance:} Selecting samples that show the most variation across traces
                \item \textbf{Sum of Squares of t-difference (SOST):} Identifies samples where the difference in means between classes (e.g., key hypotheses) is most statistically significant
                \item \textbf{Principal Component Analysis (PCA):} A dimensionality reduction technique that transforms original, correlated variables into a new set of uncorrelated variables (principal components), ordered by the amount of variance they explain
            \end{itemize}
        \end{itemize}
    \end{block}
\end{frame}


\begin{frame}
    \frametitle{Sum-of-Differences}

    Instead of relying on complex statistics, we can use a simple and intuitive method to find the best Points of Interest (POIs).

    \begin{block}{}
        Identify the exact moments in time where the side-channel traces for different key hypotheses are \textbf{most different} from each other.
    \end{block}

    This method involves three main steps, which we will now cover one by one.

\end{frame}

\begin{frame}
    \frametitle{Feature Selection: The Sum-of-Differences Method}
    
    A straightforward way to find the most interesting points in a trace is to find where the average traces for different key hypotheses \textbf{differ the most}.
    
    \begin{block}{Step 1: Calculate Average Traces}
        For each key hypothesis $k$ (or leakage model group), calculate its average trace, $M_k$. For each sample point $i$ in the trace, this is:
        \[ M_{k,i} = \frac{1}{T_k} \sum_{j=1}^{T_k} t_{j,i} \]
        where $T_k$ is the number of traces for hypothesis $k$.
    \end{block}
\end{frame}
    

\begin{frame}{Feature Selection: The Sum-of-Differences Method}
    \begin{block}{Step 2: Create a Master Difference Trace}
        Create a single "master trace," $D$, by summing the absolute differences between all pairs of the average traces at each sample point $i$:
        \[ D_i = \sum_{k_1, k_2} |M_{k_1, i} - M_{k_2, i}| \]
    \end{block}
    
\end{frame}

\begin{frame}
    \frametitle{From Difference Trace to Points of Interest}
    
    The master difference trace, $D$, will have large peaks at the time samples where the leakage is most pronounced. These peaks are our candidate Points of Interest (POIs).
    
    % \includegraphics[width=0.8\textwidth]{difference_trace_example.png}
    
    \begin{alertblock}{Step 3: Pruning the Peaks}
        It is crucial to select points that are not only significant but also \textbf{separated in time}. This ensures they capture different aspects of the computation and improves the stability of the covariance matrix.
        
        \begin{itemize}
            \item A simple "pruning" method is to select a peak and then eliminate its nearest neighbors from consideration.
            \item Repeat until the desired number of POIs (e.g., 3 to 5) is reached.
        \end{itemize}
    \end{alertblock}
    
    The final set of pruned peaks gives us the vector of features to be used for building our templates.
    
\end{frame}