\section{Template Attacks} 
\begin{frame}
    \frametitle{Profiled Side-Channel Attacks (SCAs)}
    \begin{block}{Overview}
        \begin{itemize}
            \item A powerful class of SCAs designed to attack targets with minimal measurements (even down to a single one).
            \item Overcome countermeasures that restrict the number of measurements an attacker can acquire.
            \item Involve two main phases:
                \begin{enumerate}
                    \item \textbf{Profiling Phase:} Attacker builds an accurate model of the device's side-channel behavior.
                    \item \textbf{Exploitation Phase:} The model is used to extract the unknown key from the target device.
                \end{enumerate}
        \end{itemize}
    \end{block}
\end{frame}


\begin{frame}
    \frametitle{Profiled SCAs: The Template Attack Advantage} % Improved Title
    \begin{block}{The Profiling Advantage}
        \begin{itemize}
            \item Performed on a device \textit{identical} to the target, but fully controlled by the attacker.
            \item This allows setting the secret key value at will and performing unlimited measurements.
            \item Enables deriving a "perfect" side-channel behavior or "profile" for every possible value of the key bits.
        \end{itemize}
    \end{block}
    \begin{block}{Core Idea of Template Attacks}
        \begin{itemize}
            \item TAs are considered the \textbf{most effective profiled attack }in an information-theoretic sense.
            \item They can, in principle, defeat masking and hiding countermeasures if enough measurements are collected during profiling.
        \end{itemize}
    \end{block}
\end{frame}



\begin{frame}
    \frametitle{Template Attacks: Statistical Modeling} 
    \begin{block}{Modeling Side-Channel Traces}
        \begin{itemize}
            \item Each side-channel measurement trace is assumed to be affected by \textbf{additive Gaussian noise}.
            \item Each trace $\hat{T}^{(k_i)}$ (for a specific key bit $k_i$) is modeled as a random vector variable $T^{(k_i)}$.
            \item $T^{(k_i)}$ is assumed to follow a \textbf{multivariate Gaussian distribution}: $\mathcal{N}(\mu^{(k_i)}, \Sigma^{(k_i)})$.
                \begin{itemize}
                    \item $\mu^{(k_i)}$: Mean vector (average trace for key $k_i$).
                    \item $\Sigma^{(k_i)}$: Covariance matrix (describes how data points vary together for key $k_i$).
                \end{itemize}
            \item The mean vector ($\mu$) and covariance matrix ($\Sigma$) are computed over selected \textbf{points of interest} (POIs) in the trace, where significant leakage occurs.
        \end{itemize}
    \end{block}
\end{frame}



\begin{frame}
    \frametitle{Template Attack: Multivariate Gaussian PDF}
    \begin{block}{The Probability Density Function (PDF)}
        For a trace $x$ (composed of selected POIs) belonging to a key hypothesis $k_i$, its probability density function is given by:
        $$
        \text{Pr}(T^{(k_i)}=x) = \frac{\exp\left(-\frac{1}{2}(x-\mu^{(k_i)})^T(\Sigma^{(k_i)})^{-1}(x-\mu^{(k_i)})\right)}{\sqrt{(2\pi)^n \det(\Sigma^{(k_i)})}}
        $$
        \begin{itemize}
            \item $n$: Number of selected points of interest (dimension of the trace vector $x$).
            \item $\mu^{(k_i)}$: Mean vector for key hypothesis $k_i$.
            \item $\Sigma^{(k_i)}$: Covariance matrix for key hypothesis $k_i$.
        \end{itemize}
    \end{block}
\end{frame}




\begin{frame}
    \frametitle{Template Attack: Profiling Phase - Building Templates} 
    \begin{block}{Generating Templates}
        \begin{itemize}
            \item The attacker, on their controlled device, sets a specific key bit value $k_i$ (e.g., $k_i=0$ or $k_i=1$).
            \item For each possible value of $k_i$, a large number of side-channel traces are collected (from the POIs).
            \item From these measurements, \textit{sample estimates} of the mean vector ($\hat{\mu}^{(k_i)}$) and covariance matrix ($\hat{\Sigma}^{(k_i)}$) are derived.
            \item These statistical models ($\hat{\mu}^{(k_i)}, \hat{\Sigma}^{(k_i)}$) form the \textbf{templates} for each key bit value.
        \end{itemize}
    \end{block}
\end{frame}


\begin{frame}
    \frametitle{Template Attack: Profiling Phase}
    \begin{block}{Example: Targeting a Key Portion}
        \begin{itemize}
            \item If targeting an 8-bit key byte, an attacker might build $2^8 = 256$ templates (one for each possible byte value).
            \item Alternatively, using a divide-and-conquer approach, they could target each bit individually, building two templates (for '0' and '1') for each bit.
        \end{itemize}
    \end{block}
\end{frame}

\begin{frame}
    \frametitle{Optimization: Grouping by Leakage Model}
    
    Instead of modeling the raw key value, we can build templates based on a chosen \textbf{leakage model}. This groups key values that are expected to produce similar side-channel leakage.
    
    \begin{block}{Common Leakage Models for Grouping}
        \begin{itemize}
            \item \textbf{Hamming Weight (HW) Model:} Groups keys based on the number of '1's in a sensitive intermediate value (e.g., an S-box output). This reduces 256 key values to just 9 groups.
            
            \item \textbf{Hamming Distance (HD) Model:} Groups keys based on the number of bits that flip between two consecutive states.
            
            \item \textbf{Other models} can also be used, depending on the device's specific physical characteristics.
        \end{itemize}
    \end{block}
    
    \begin{alertblock}{}
        By grouping key hypotheses that are "close" in the leakage space, we drastically reduce the number of templates that need to be built, making the profiling phase far more practical.
    \end{alertblock}
    
\end{frame}

\begin{frame}
    \frametitle{Building the Template: The Mean Vector}
    
    Once the Points of Interest (POIs, more on them later) have been selected, we can build the statistical model for each key hypothesis $k$.
    
    \begin{block}{Step 1: Calculate the Mean Vector ($\hat{\mu}^{(k)}$)}
        \begin{itemize}
            \item For a given key hypothesis $k$, collect a large number of profiling traces ($T_k$).
            \item For each selected POI $s_i$, calculate the average power consumption across all $T_k$ traces:
            \[ \hat{\mu}_i^{(k)} = \frac{1}{T_k} \sum_{j=1}^{T_k} t_{j, s_i} \]
            where $t_{j, s_i}$ is the measurement at POI $s_i$ in trace $j$.
            
            \item The mean vector is simply the collection of these average values for all POIs.
            \[ \hat{\mu}^{(k)} = [\hat{\mu}_1^{(k)}, \hat{\mu}_2^{(k)}, \dots, \hat{\mu}_n^{(k)}]^T \]
        \end{itemize}
    \end{block}
  
\end{frame}

\begin{frame}
    \frametitle{Building the Template: The Covariance Matrix}

    \begin{block}{Step 2: Calculate the Covariance Matrix ($\hat{\Sigma}^{(k)}$)}
        \begin{itemize}
            \item The covariance matrix captures how the POIs vary together.
            \item For each pair of POIs ($s_i, s_{i'}$), calculate the covariance:
            \[ c_{i, i'}^{(k)} = \frac{1}{T_k-1} \sum_{j=1}^{T_k} (t_{j, s_i} - \hat{\mu}_i^{(k)})(t_{j, s_{i'}} - \hat{\mu}_{i'}^{(k)}) \]
            \item The full $n \times n$ covariance matrix is constructed from these values. The diagonal elements ($c_{i,i}$) are the variances of each individual POI.
            \[ \hat{\Sigma}^{(k)} = \begin{pmatrix} c_{1,1} & c_{1,2} & \dots \\ c_{2,1} & c_{2,2} & \dots \\ \vdots & \vdots & \ddots \end{pmatrix} \]
        \end{itemize}
    \end{block}
    
    The pair $(\hat{\mu}^{(k)}, \hat{\Sigma}^{(k)})$ forms the complete template for key hypothesis $k$.
    
\end{frame}




\begin{frame}
    \frametitle{Temlate Attacks: Matching Phase}
    \begin{block}{Key Recovery Steps}
        \begin{itemize}
            \item A single (or few) side-channel trace(s) $\hat{T}$ is acquired from the target device, where the key value is unknown
            \item Measurement conditions must be identical to the profiling phase
            \item For each possible key bit value, the likelihood of the acquired trace $\hat{T}$ belonging to that template's distribution is evaluated
            \item This is done by calculating the \textbf{a-posteriori probability} $\text{Pr}(k_i | \hat{T})$, typically using Bayes' theorem:
            $$
            \text{Pr}(k_i | \hat{T}) = \frac{\text{Pr}(\hat{T} | k_i) \cdot \text{Pr}(k_i)}{\sum_j \text{Pr}(\hat{T} | k_j) \cdot \text{Pr}(k_j)}
            $$
            \item The key bit value $k_i$ that maximizes this a-posteriori probability is selected as the most likely secret key bit.
        \end{itemize}
    \end{block}
\end{frame}
\begin{frame}
\frametitle{Template Attacks}
    \begin{block}{Statistical Efficacy}
        \begin{itemize}
            \item The assumption of multivariate Gaussian distributions for power consumption has proven to be a very good practical approximation.
            \item Statistical tests are used to determine the "goodness of fit" of a model to the actual device behavior, accounting for noise.
        \end{itemize}
    \end{block}
\end{frame}