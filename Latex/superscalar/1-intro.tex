
\section{Introduction \& Context}

%---------------------------------------------------------
%Changing visivility of the text
% slide 1
%\begin{frame}
%\frametitle{The Evolving Target: Complex CPUs}
%\begin{block}{Remark}

%\end{block}

%\begin{itemize}
%    \item<1-> Text visible on slide 1
%    \item<2-> Text visible on slide 2
%    \item<3> Text visible on slides 3
%    \item<4-> Text visible on slide 4
%\end{itemize}
%\end{frame}


\begin{frame}
    \frametitle{Recap: What are Side-Channel Attacks?}
    \begin{block}{Definition}
        \begin{itemize}
            \item Side-Channel Attacks (SCAs) exploit unintended physical emanations (e.g., power consumption, electromagnetic radiation, execution time) from a device
            \item These emanations are correlated with the secret data being processed, allowing an attacker to infer sensitive information (e.g., cryptographic keys)
        \end{itemize}
    \end{block}

    \begin{block}{The Fundamental Question }
        \begin{itemize}
            \item How do we accurately \textit{model} these physical emanations to find the leaked information?
            %\item This presentation delves into the specifics of leakage modeling, especially on modern, complex CPUs.
        \end{itemize}
    \end{block}
\end{frame}
%---------------------------------------------------------
%---------------------------------------------------------
\begin{frame}
    \frametitle{Complex CPU Targets}
    \begin{block}{SCA Challenges on Modern CPUs}
    \begin{itemize}
        \item SCAs are increasingly targeting complex platforms:
            \begin{itemize}
                \item SCA targets have evolved from simple micro-controllers and smart-cards to complex single-core and laptop-grade CPUs
                \item Now, side-channel attacks are also relevant for superscalar CPUs
            \end{itemize}
        \item Assessing vulnerability and validating countermeasures becomes significantly more difficult with increasing target complexity
    \end{itemize}
    \end{block}
\end{frame}


\begin{frame}
    \frametitle{Why Microarchitecture Matters for SCA}
    \begin{block}{Main Idea} 
        \begin{itemize}
            \item To assess the side-channel vulnerability of software on a CPU, we \textbf{must} consider the CPU's microarchitectural features.
            \item Correct execution only requires the object code to match the CPU at ISA level.
            \item However, the extent of side-channel leakage also depends on the processor's microarchitecture.
        \end{itemize}
    \end{block}
\end{frame}

\begin{frame}
    \frametitle{Why Microarchitecture Matters for SCA}
    \begin{block}{Dangers of Ignoring Microarchitecture}
        \begin{itemize}
            \item Side-channel countermeasures, if designed without microarchitectural awareness, can be invalidated.
            \item \textbf{Example}: Accidental value combinations due to register reuse, despite careful assembly implementations.
        \end{itemize}
    \end{block}
\end{frame}
%---------------------------------------------------------

%---------------------------------------------------------
