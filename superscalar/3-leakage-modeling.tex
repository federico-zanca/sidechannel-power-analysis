\section{SCA Characterization \& Leakage Modeling}

\begin{frame}
    \frametitle{Characterizing Microarchitecture Leakage}
    \begin{block}{The Foundation of Leakage}
        \begin{itemize}
            \item Gates driving large capacitive loads are primary sources of side-channel leakage.
            \item Power consumption in such scenarios is well-modeled by the \textbf{Hamming distance} of two values asserted on their outputs in subsequent clock cycles
        \end{itemize}   
    \end{block}

    \begin{block}{Detecting Leakage}
        \begin{itemize}
            \item \textbf{Pearson's correlation coefficient} is used to statistically compare measured power consumption with the predicted leakage model
            \item A correlation statistically distinguishable from zero (with >99.5\% confidence) in the correct clock cycle indicates a leakage
        \end{itemize}
    \end{block}
\end{frame}


\begin{frame}
    \frametitle{Leakage Measurement - Cortex-A7}
    \begin{block}{Data Collection}
        \begin{itemize}
            \item Seven micro-benchmarks: small (2-4) instruction sequences designed to trigger specific component activities
            \item Ran with randomly generated values at each execution, triggered by GPIO
            \item Cache effects eliminated by measuring executions after the first one and inserting 100 \texttt{nop}s
            \item Data acquired: \textbf{100,000 power traces} per benchmark
            \item Each trace was an average of 16 individual executions with the same input, reducing noise
            \item Register File (RF) leakage was evaluated separately by pre-charging destination registers
        \end{itemize}
    \end{block}
\end{frame}

%------------------------------------------------------

\begin{frame}
    \frametitle{Observed Leakage: Registers}
    \begin{block}{Register File (RF)}
        \begin{itemize}
            \item No statistically significant leakage observed from RF read-ports.
            \item Attributed to a short capacitive load, as Issue Stage (IS) buffers drive execution units
        \end{itemize}
    \end{block}
\end{frame}

\begin{frame}
    \frametitle{Observed Leakage: IS/EX}
    \begin{block}{Issue/Execution (IS/EX) Buffers}
        \begin{itemize}
            \item Outputs show significant leakage
            \item \textbf{Modeled by Hamming Distance:} between values of a source operand of an older and a younger single-issued instruction
                \begin{itemize}
                    \item Leakage prominent when operands share the same bus (same source operand position)
                \end{itemize}
            \item \textbf{Hamming Weight Leakage:} Also observed when \texttt{mov}s are interleaved with \texttt{nop}s (due to \texttt{nop} implementation with zero-valued operands)
            \item Dual-issued arithmetic instructions show no measurable leakage among source operands (no shared resources before computation)
        \end{itemize}
    \end{block}
\end{frame}



\begin{frame}
    \frametitle{Observed Leakage: ALUs }
    \begin{block}{ALU and Shift Buffer}
        \begin{itemize}
            \item \textbf{ALU:} Leakage dependent on the \textbf{Hamming weight} of the instruction result
                \begin{itemize}
                    \item Inferred due to ALUs asserting results on previously zero-precharged signals
                \end{itemize}
            \item \textbf{Shift Buffer:} A small leakage proportional to the \textbf{Hamming weight} of the shifted value is present
        \end{itemize}
    \end{block}
\end{frame}

\begin{frame}
    \frametitle{Observed Leakage: EX/WB Buffers}
    \begin{block}{Execution/Write-Back (EX/WB) Buffers}
        \begin{itemize}
            \item Leakage mirrors IS/EX buffers.
            \item \textbf{Modeled by Hamming Distance:} between the \textit{results} of subsequent single-issued instructions.
                \begin{itemize}
                    \item Occurs regardless of destination register sharing or data-flow relationship
                \end{itemize}
            \item \textbf{Hamming Weight Leakage:} Also present for instruction results.
                \begin{itemize}
                    \item Attributed to \texttt{nop} instructions resetting the WB bus to zero 
                \end{itemize}
        \end{itemize}
    \end{block}
\end{frame}




\begin{frame}
    \frametitle{Observed Leakage: MDR}
    \begin{block}{Memory Data Register (MDR) and Align Buffer}
        \begin{itemize}
            \item Potential leakage source during load/store instruction sequences
            \item \textbf{Modeled by Hamming Distance:} between two subsequently loaded or stored values
            %\item Applies even to half-word and single-byte loads/stores (power proportional to HD between full 32-bit words).
            \item Suggests a separate buffer in the Load Store Unit (LSU) for sub-word realignment
            \item Presence of this \textbf{align buffer} and its leakage was confirmed experimentally
        \end{itemize}
    \end{block}
\end{frame}


\begin{frame}
    \frametitle{Superscalar Leakage Modeling}
    \begin{block}{Leakage from Algorithmically Independent Instructions}
        \begin{itemize}
            \item Observed information leakage that combines values from potentially independent instructions, driven by four causes:
                \begin{enumerate}
                    \item Instruction scheduling order
                    \item Position of source operands
                    \item Single or dual-issuing of instructions
                    \item Potential data remanence in LSU buffers
                \end{enumerate}
        \end{itemize}
    \end{block}
\end{frame}
\begin{frame}
    \frametitle{Superscalar Leakage Modeling}
    \begin{block}{Subtle Code Changes can be Harmful}
        \begin{itemize}
            \item Even apparently harmless changes to assembly code (e.g., swapping source operands of a commutative operation like XOR) can lead to side-channel leakage
            \item This is due to altered pipeline resource sharing
            \item Such changes are difficult to detect by tools focusing only on instruction semantics or manual audits
        \end{itemize}
    \end{block}
\end{frame}




\begin{frame}
    \frametitle{Superscalar Leakage: Vulnerabilities}
    \begin{block}{Impact of Dual-Issuing}
        \begin{itemize}
            \item Worsens effects of instruction scheduling and operand position.
            \item Leakage can stem from combinations of source operands of \textit{non-consecutive} instructions if an intermediate instruction is dual-issued
            \item Highlights the crucial, often-neglected importance of instruction scheduling for preventing side-channel leakage
            \item Dual-issuing could also potentially \textbf{enhance security} by enabling parallel computation of two shares in masking schemes
        \end{itemize}
    \end{block}
\end{frame}
\begin{frame}
\frametitle{Superscalar Leakage: Vulnerabilities}
    \begin{block}{Data Remanence and NOPs}
        \begin{itemize}
            \item \textbf{Data Remanence:} In MDR and LSU buffers, old data can accidentally combine with current computation results, creating harmful leakage.
            \item \textbf{NOP Operations:} While semantically neutral, \texttt{nop} instructions can introduce new leakage modes (e.g., by resetting buses to zero). They are not \textit{security neutral}
        \end{itemize}
    \end{block}
\end{frame}